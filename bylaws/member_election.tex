\section{Process} Each Fall and Winter semester, the procedure to be conducted for the election of student members is as follows:
\begin{enumsubsection}
\itemnotoc  The President will request a list of scholastically eligible students from the Registrar  or their representative.
\itemnotoc The officers will prepare the list of scholastically eligible students. They will remove the names of all initiated members from the list, as well as others who do not meet the criteria set forth in Article IV.2 of the chapter Constitution. The Secretary will simultaneously prepare the Report of Eligibility for forwarding to the National Headquarters.
\itemnotoc The officers will send a letter of invitation and an information packet to all remaining scholastically eligible undergraduate students.
\itemnotoc Eligible students will receive letters from the President and pertinent information from their respective Vice President. % Added F23
%\itemnotoc Letters from the Dean of the College of Engineering will be sent to each eligible undergraduate student and to the parents of each eligible undergraduate student.  Eligible graduate students will receive invitations and information from the Graduate Student Vice President. % Struck in F23 2nd actives
\itemnotoc Each prospective new member must be assessed for exemplary character as defined by the officers in Appendix \ref{sec:ugradreqs} or \ref{sec:gradreqs} as appropriate. %ref
\itemnotoc  Each electee member must complete the minimum set of requirements as specified in Appendix \ref{sec:ugradreqs} or \ref{sec:gradreqs} as appropriate. Where circumstances merit, the Vice President or Graduate Student Vice President may choose to allow any electee to substitute one form of requirement for another, provided that the total number of hours completed is not diminished as a result of this. In exceptional circumstances, the Advisory Board may, by a \nicefrac{5}{7} vote, waive any such requirement for any electee.

\itemnotoc Two election of candidates meetings must be held prior to initiation.
\begin{enumsubsubsection}
\itemnotoc The first meeting (``Second Actives") focuses on the character of the candidates for membership and must be held following the character interviews described in Appendix \ref{sec:ugradreqs} or \ref{sec:gradreqs} as appropriate. %link
In a closed session, the active members  hold a discussion and vote on each prospective new member's character, the outcome of which must be reported to the candidates and to the second election of candidates meeting, as a recommendation on how to consider the character of those assessed.  Those candidates recommended by the Vice President or the Graduate Student Vice President may be considered as a group.  Active members will be given the opportunity to remove any questionable candidates from the group.  All questionable or not-recommended candidates must be voted on individually.  To be recommended for election, a candidate must receive a three-fourths (3/4) vote of the active membership present at the meeting.   Active members, whether undergraduate or graduate students, are eligible to vote on new members.% Alumnus members are not eligible to vote on new members.
\itemnotoc The second election of candidates meeting (``Third Actives") takes place approximately 4 weeks prior to initiation (or enough time as is needed to report the confirmed electees to headquarters) and focuses on the candidates' progress toward completion of the requirements set forth in Appendix \ref{sec:ugradreqs} or \ref{sec:gradreqs} as appropriate. %link
 Candidates whose character was not discussed at Second Actives must have their character reviewed at this meeting. In a closed session, the active members hold a discussion and vote on whether to elect each candidate to membership.  To be elected, a candidate must receive a three-fourths (3/4) vote of the active membership present at the meeting.   Active members, whether undergraduate or graduate students, are eligible to vote on new members. Alumni members are not eligible to vote on new members.
\end{enumsubsubsection}
At each of these meetings, the Advisory Board may deactivate any member as specified in the  \href{http://www.tbp.org/off/ConstBylaw.pdf}{Tau Beta Pi Association Bylaws Section 6.06}; criteria include non-attendance at that meeting. Unless otherwise specified by the Advisory Board,
members so deactivated are automatically reactivated at the end of the meeting.
\itemnotoc Any candidate who is not elected to membership is not eligible to initiate that semester.
\itemnotoc  The Vice President or the Graduate Student Vice President must announce either election/recommendation or rejection to the prospective new members following each election of candidate meeting.

%\itemnotoc Each questionable  electee will be reviewed and may be dismissed from the electee class by a three-fourths (3/4) vote of the Officer Corps.
\itemnotoc Initiation must be held after all other general meetings and before the end of classes each term. Alternate initiations may be held when needed at the discretion of the officer corps with approval of the Advisory Board.
\end{enumsubsection}
\section{Alumni and Eminent Engineer Candidates} Election of suitable alumni members and eminent engineer candidates is encouraged by this chapter. Any active member may present to the officer corps nominees for alumni membership prior to the first election of candidates meeting. The nominees' resumes must be presented. At the second election of candidates meeting, the active membership can elect any or all of the nominees.