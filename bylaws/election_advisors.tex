\section{Election Meeting, Quorum,  Eligibility, and Nominations} Election of advisors will take place at the officer election meeting, under the same quorum,  active status, and nomination procedures  as described in Bylaw~\ref{sec:oficElec}. Anyone meeting the criteria in Bylaw~\ref{sec:advisors} is eligible to run for an advisor position.
\section{Candidate Speeches} Each advisor nominee may give a speech to, and answer questions from, the general assembly. All nominees are limited to 1.5 minutes for both speech and questions and answers. These limits may be extended by motion from the floor.
\section{Candidate Discussion} Following each speech, the general assembly will engage in discussion concerning the candidate. Since each advisor is treated as its own office, in the event that multiple advisors pursue election in the same term, it is not necessary that anyone but the candidate in question leave the room for discussion. This discussion must remain strictly confidential and is limited to matters pertaining to an individual's ability to successfully carry out the responsibilities of the office. 
\section{Winning the Election} Each candidate that receives a majority of the votes cast (not counting abstentions as a vote for or against) will be elected to office. The votes must be counted by three individuals who must be either advisors or officers not pursuing election that semester.
\section{Term Length} \label{sec:advisorLength} Advisor term length will be subject to motion from the floor as part of the election proceedings. An advisor candidate must indicate their preferred term length, and cannot be elected to a term longer than that. The maximum length for an individual term is three years. The term length is considered part of the motion to elect and therefore may be amended by the body. If an advisor fails to receive a majority of votes cast, a motion may be made to elect them to a shorter term, to a minimum of one semester.
\section{Term Commencement} Advisor terms begin on the day following the chapter's main initiation directly following their election.