%%Officer Corps Changeable
\chapter{Undergraduate Electee Requirements}\label{sec:ugradreqs}
\section{Meetings} Each electee must attend all meetings as specified in Bylaw~\ref{sec:genmeetings}, except the general actives meeting and the election of candidates  meetings. A missed meeting may be made up with one extra hour of service that is approved by the Vice President. 
With the exception of Election of Officers meeting, a missed meeting may alternatively be made up by attending an extra social event approved by the Vice President. Given the typical length of the Officer Elections Meeting, electees who stay for the full duration will be awarded 2 service hours. At the two hour mark attendance will be retaken and those who remain will be awarded 1 service hour, those who stay until the end will be awarded 2.
\section{Character} Each electee will be assessed for exemplary character. This will consist of filling out a student information survey with questions focusing on exemplary character, and attending  an interview or interviews with active members as specified by the Vice President. % two interviews, not longer than thirty minutes each, each given by at least one active member. %removed F17
\section{Dues} Each electee must pay the one-time-only membership dues as set by the Officer Corps. Any electee who is unable to pay the fee may submit a chapter fee waiver form or apply for a National Initiation-Fee Loan. The form responses will be viewed only by the President, Vice Presidents, and Treasurer, who will review them and grant the waiver as they see fit.
\section{Electee Exam} Each electee must complete an electee exam that is written by the Vice President with input from the officers. This exam will include a section about the history of Tau Beta Pi, a section about the chapter's officers, and a signature form, requiring at least two signatures from chapter officers, and at least five signatures from active Tau Beta Pi members who are not officers.
\section{Peer Interviews} Each electee must conduct interviews of six fellow Tau Bates, of which at least two must be active members. The interviews will consist of six creative questions. 
%\section{Career Fair}\label{ugrad:CF} In the fall terms each electee must complete two hours of service for Career Fair in jobs specified by the External Vice Presidents. 
\section{Tutoring} Each electee must complete at least one hour of university tutoring. %\added{The tutoring requirement may be satisfied either by flyering for tutoring or by serving as a tutor.}
\section{K-12 Outreach} \label{ugrad:MindSET} Each electee must participate in a minimum of 4 hours of approved K-12 Outreach events as determined by the K-12 Outreach Officers. % Prior Language: at least one approved K-12 Outreach event for a minimum of 3.5 hours as determined by the K-12 Outreach Officers. % Was one event prior to F20 (COVID) and 3.0 hours prior to W21 (COVID) % Was 3.5 hours going into F23. Changed to 4 hours.
\section{Service} Each electee must complete at least 11.5 additional hours in the Winter through at least two different service projects offered by the chapter. Any service completed in the events listed in \ref{ugrad:CF}--\ref{ugrad:MindSET} above the minimum may be used to satisfy this requirement.  Service performed for the completion of requirements of another society will not be accepted. Extra service hours can be used to make up missed meetings. % was ten for fall prior to F20 (COVID) and twelve for winter prior to W21 (COVID)  # Was 10 and 12 prior to F22; adjusted such that the totals are 16 and ... % Was 11.5 and 9.5 going into F23

%F23, removed CF explicit requierments, as stated: Each electee must complete at least 9 additional hours of service in the Fall, and 11 additional hours in the Winter through at least two different service projects offered by the chapter. Any service completed in the events listed in \ref{ugrad:CF}--\ref{ugrad:MindSET} above the minimum may be used to satisfy this requirement.  Service performed for the completion of requirements of another society will not be accepted. Extra service hours can be used to make up missed meetings.

\section{Professional Development}\label{sec:ugpdreq} Each electee must complete 2 hours of professional development including but not limited to participating in Engineering Futures and TBP sponsored Information Sessions. The Vice President and Service Coordinator may, on a case by case basis, allow up to 2 service hours to fill this requirement. % F23 removed: North Campus Sustainability Hours, chapter sponsored MLK Luncheons, TBP Tech Talks,

\section{Socials}\label{sec:ugsocialreq} Each electee must attend at least two social events sponsored by Tau Beta Pi. Extra social  events can be used to make up missed meetings (except Officer Elections). For Winter 2021, social and professional development activities shall satisfy the requirements for Sections \ref{sec:ugpdreq} and \ref{sec:ugsocialreq}. All undergraduate electees must have, at a minimum, one credit from both Section \ref{sec:ugpdreq} and \ref{sec:ugsocialreq} that do not double count.
\section{Group Meetings} \label{group_meetings} Each electee must attend three group activities with at least half of their electee group present. One of the three electee group activities must occur with at least half of another electee group present. All electee group activities must occur before the fifth general meeting of a semester.  This can be in the form of a social or service activity but does not have to be a Tau Beta Pi sponsored event. A missed group meeting may be made up with one extra hour of service or an extra social that is approved by the Vice President.
\section{Initiation} Each electee must attend initiation at the end of the term. This is an absolute requirement and cannot be made up. An electee who has completed all other requirements but misses initiation will not become a member of Tau Beta Pi. Such an electee may either attend the initiation of another chapter or wait until the following term to attend initiation.
%\section{MENTORSHIP PROGRAM} Each electee must participate in the chapter mentorship program. %\removed{Following the First General meeting, each electee must fill out a survey that will be used to match them with a mentor. After being matched, the electee must meet with their mentor at least once before initiation.} % Removed in F20 (COVID) % Full section removed in F23

\section{Timing and Deadlines}  The Vice President, with input from the Officer Corps, will establish appropriate deadlines for the requirements each semester. An electee may choose to count any activities completed after the final deadline but before the end of the term toward the following semester, or, if approved by the Vice President, the current term.

%%Officer Corps Changeable
\chapter{Graduate Electee Requirements}\label{sec:gradreqs}%%In much need of updating
\section{Meetings} Each electee must attend all meetings as specified in Bylaw~\ref{sec:genmeetings}, except the general actives meeting and the election of candidates  meetings. A missed meeting may be made up with one extra hour of service that is approved by the Grad Student Vice President. 
With the exception of Election of Officers meeting, a missed meeting may alternatively be made up by attending an extra social event approved by the Graduate Student Vice President. Given the typical length of the Officer Elections Meeting, electees who stay for the full duration will be awarded 2 service hours. At the two hour mark attendance will be retaken and those who remain will be awarded 1 service hour, those who stay until the end will be awarded 2.
\section{Character} Each electee will be assessed for exemplary character. This
will consist of filling out a student information survey with questions focusing on exemplary character.
\section{Service}	In addition to the requirements stated in Constitution Article IV.2.b, the electee must complete no less than ten service hours.  Electees may complete up to three service hours on their own if they provide adequate proof of their involvement. Service performed for the completion of requirements of another society will not be accepted. % was 10 prior to F20 (COVID) and 5 during covid
\section{Interview}	The electee must undergo an interview by the Graduate Student Vice President and/or at least one active member. It is recommended that each interviewer be a graduate student. 
\section{Advisor Form}	Each electee must get the standard advisor recommendation form signed by their advisor or graduate program coordinator and turned into the Graduate Student Vice President by their interview.
\section{Dues}	Each electee must pay the one-time only membership dues as set by the Officer Corps. Any electee who is unable to pay the fee may submit a chapter fee waiver form or apply for a National Initiation-Fee Loan. The form responses will be viewed only by the President, Vice Presidents, and Treasurer, who will review them and grant the waivers as they see fit.
\section{Socials}\label{sec:grsocialreq}	Each electee must attend at least two social  or professional development events sponsored by Tau Beta Pi.  At least one of these social events must be a graduate student social as designated by the Graduate Student Vice President. % Was 2 prior to F20 (COVID) and will be 3 as of 1/1/22.
\section{Group Meetings} \label{grad_group_meetings} Each electee must attend two group activities with at least half of their electee group present. One electee group activity must occur before the election of candidates, and the other before the fifth general meeting. This can be in the form of a social or service activity but does not have to be a Tau Beta Pi sponsored event. A missed group meeting may be made up with one extra hour of service or an extra social that is approved by the Graduate Student Vice President. For Fall 2020, group meetings and social activities shall satisfy the requirements for Sections \ref{sec:grsocialreq} and \ref{grad_group_meetings}.

\section{Initiation}	Each electee must attend initiation at the end of the term. This is an absolute requirement and cannot be made up. An electee who has completed all other requirements but who misses initiation may not become a member of Tau Beta Pi. Such an electee may either attend the initiation of another chapter or must wait until the following term to attend the initiation.  
\section{Timing and Deadlines}  The Graduate Student Vice President, with input from the Vice President, will establish appropriate deadlines for the graduate student requirements each semester. An electee may choose to count any activities completed after the final deadline but before the end of the term toward the following semester, or, if approved by the Graduate Student Vice President, the current term.

%%Officer Corps Changeable
\chapter{Alumni Candidate Requirements}\label{sec:alumnireqs}%%In much need of updating
\section{Scope} The contents of this appendix apply only to those alumni that were eligible to elect during their final semester in school as either an undergraduate or graduate student and that are not full-time students at the  University of Michigan. Alumni candidates who are full-time students at the University of Michigan shall be subject to the appropriate requirements in Appendix \ref{sec:ugradreqs} or Appendix \ref{sec:gradreqs} dependent on their current status as a student.

\section{Chapter Involvement} Each candidate will complete a total of 7 hours of TBP activities. A minimum of one hour must come from a TBP service event.
\section{Interview} Each candidate must undergo a casual interview by the Graduate Student Vice President and/or at least one active member. %F23, officer corps changes Membership Officer to GSVP

\section{Character} Each candidate will be assessed for exemplary character. This will consist of filling out a student information survey with questions focusing on exemplary character.
\section{Dues} Each candidate must pay the one-time only membership dues as set by the Officer Corps. 
\section{Initiation} Each candidate must attend initiation at the end of the term. This is an absolute requirement and cannot be made up. An candidate who has completed all other requirements but who misses initiation may not become a member of Tau Beta Pi. Such an electee may either attend the initiation of another chapter or must wait until the following term to attend the initiation.  
\section{Timing} Each alumni candidate will be considered active in a given semester when active progress is made towards the initiation requirement. Active progress is defined as completing two hours in a given academic semester.
%\section{MENTORSHIP PROGRAM} Each electee must participate in the chapter mentorship program, either as a mentor to an undergraduate electee or as a mentee to an active graduate student. Following the First General meeting, each electee must fill out a survey that will be used to match them with a mentor/mentee. After being matched, the electee must meet with their mentor/mentee at least once before initiation. % Removed F23

\section{Deadlines}  The Graduate Student Vice President, with input from the Vice President, will establish appropriate deadlines for the alumni requirements each semester. An electee may choose to count any activities completed after the final deadline but before the end of the term toward the following semester, or, if approved by the Graduate Student Vice President, the current term. %Pre F23 text: The Membership Officer, with input from the Vice President and Graduate Student Vice President, will establish appropriate deadlines for the alumni requirements each semester. An electee may choose to count any activities completed after the final deadline but before the end of the term toward the following semester, or, if approved by the Membership Officer, the current term.


%+membership fee
%+casual interview
%suggest them to attend 4 hours of service, 1 social, and 2 chapter meetings to add up to 7 hours



%%Officer Corps Changeable
\chapter{Distinguished Active Status Guidelines}\label{sec:DAstatus}
\section{Requirements} Members may achieve Distinguished Active status by completing at minimum the following requirements in the course of one term (Fall or Winter Terms):
\begin{enumsubsection}
\item{Leadership:} Serve as a group leader, project leader, committee chair,  officer/advisor, or achieve a sufficient level of involvement in a committee as defined in Appendix~\ref{sec:commPart}.
\item{Interviews:} Conduct one interview in the exemplary character assessment process of electees  Graduate students may fulfill this requirement by assisting in Graduate Student Interviews.
\item{Voting Meeting Attendance}\label{sec:DAvotingMeeting} Attend both election of candidate meetings and the election of officers meeting (Second and Third Actives and Elections). If missed, each of these meetings must be made up with an hour of service. Given the typical length of the Officer Elections Meeting, those who stay for the full duration will be awarded 2 service hours. At the two hour mark attendance will be retaken and those who remain will be awarded 1 service hour, those who stay until the end will be awarded 2.
\item{Meeting Attendance} Attend two general meetings in addition to the three listed in (\ref{sec:DAvotingMeeting}). These meetings, if missed, may be made up by either completing an extra hour of service or attending an extra social event.
\item{Service}\label{sec:DAservice} Complete at least eight hours of Tau Beta Pi service projects (project leaders may count only the participating hours of their project toward these hours). A single event cannot be used to simultaneously meet the leadership requirement and the entirety of the service hours requirement.
\item{Socials and Professional Development} \label{sec:DAsocial}Attend two social or professional development activities (social or professional development event leaders may count their event). In addition, electee group leaders may count additional electee group meetings (as defined in Appendix \ref{group_meetings}) as socials.
\item{Extra} Participate in either one more service hour or one more social event than required above in (\ref{sec:DAservice}) or (\ref{sec:DAsocial}).
%\item{National Convention} \removed{Serve a minimum of 3 hours volunteering at the 2017 National Convention.} 

\end{enumsubsection}
%\section{Summer DA Requirements:}
%\begin{enumsubsection}
%\item{Leadership:} Serve as a project leader, committee chair,  officer/advisor, or achieve a sufficient level of involvement in a committee as defined in Appendix~\ref{sec:commPart}.
%\item{Service}\label{sec:summerDAservice} Complete at least ten hours of Tau Beta Pi service projects (project leaders may count only the participating hours of their project toward these hours). A single event cannot be used to simultaneously meet the leadership requirement and the entirety of the service hours requirement.
%\item{Socials} \label{sec:summerDAsocial}Attend two social activities (social event leaders may count their event). 
%\item{Extra} Participate in at least 5 more hours than required above in (\ref{sec:summerDAservice}) or (\ref{sec:summerDAsocial}). Each of these hours can be earned by completing an hour of service or attending a social event.
%\end{enumsubsection}
\section{Attaining DA during Initiation Semester} An electee may count any hours beyond those needed for initiating toward achieving DA status. Requirements follow.
\begin{enumsubsection}
\item{Leadership:} Serve as a project leader, committee chair,  officer/advisor, or achieve a sufficient level of involvement in a committee as defined in Appendix~\ref{sec:commPart}. If an electee is unable to meet this requirement, an hour of service may be substituted.
\item{Service}\label{sec:electeeDAservice} Complete at least eight hours of Tau Beta Pi service projects (project leaders may count only the participating hours of their project toward these hours). 
\item{Socials} \label{sec:electeeDAsocial} Attend one social or professional development activity (social or professional development event leaders may count their event). Any electee group meetings beyond the required number can count towards this requirement.
\item{Extra} Participate in at least 3 more hours than required above in (\ref{sec:electeeDAservice}) or (\ref{sec:electeeDAsocial}). Each of these hours can be earned by completing an hour of service or attending a social event.
%\item{MENTORSHIP PROGRAM} Participate in the chapter mentorship program as a mentor. This includes filling out a matching survey at the beginning of the semester and meeting with the assigned mentee at least once before initiation. % Removed during F23

\end{enumsubsection}
\section{National Convention Distinguished Active Status} In addition, an active member may achieve Distinguished Active status by solely attending the Tau Beta Pi National Convention during that term, so long as they attend all business meetings and all other activities relevant to the chapter during the National Convention and have otherwise fulfilled all requirements to earn active status that term. % during the ... added in F23
\section{Gift} At the end of each term, in appreciation for their contribution, each Distinguished Active will be given the following benefits according the number of terms of distinguished activity:
\begin{enumsubsection}
\item{First Term:} A Tau Beta Pi stole to be worn at graduation.
%\item{Third Term:}  Invitation to attend the initiation banquet at no cost, where he/she will receive special recognition for their achievement and dedication. % Struck in F23
\item{Multiple Terms:} Additional benefits awarded for two or more  terms of Distinguished Active status will also be determined at the discretion of the Membership Officer. % F23, discretion of the Officer Corps. 
\end{enumsubsection}
\section{Timing and Deadlines to Achieve Status} Any requirements completed following initiation but before the end of term may be counted toward the current term or the following full term. In addition, any requirements completed between the winter and fall terms may be counted toward the fall. 
\section{Timing Limit to Claim} A member shall have one calendar year to claim a gift provided under this bylaw. The calendar year shall commence on the day of the initiation banquet of the term during which a status is earned.
%\section{Substitution} Where circumstances merit, the Membership Officer may choose to allow any member to substitute one form of requirement for another, provided that the total number of hours completed is not diminished as a result of this. % Prior to F23
\section{Exceptions} Where circumstances merit, the Membership Officer may choose to allow any member to substitute service hours for any form of requirement, with the exception of leadership credit, provided that the total number of hours completed is not diminished as a result of this. With consent of the Executive Committee, the Membership Officer may assign leadership credit to someone who has not otherwise earned it.

%%Officer Corps Changeable
\chapter{Prestigious Active Status Guidelines}\label{sec:PAstatus}
\section{Requirements} Members may achieve Prestigious Active status by completing at minimum the following requirements in the course of one term:
\begin{enumsubsection}
\item{DA Status:} Achieving Distinguished Active status as defined in Appendix~\ref{sec:DAstatus}.
\item{Total Involvement} Completing 32 total hours (including those from Distinguished Actives Requirements), defined as follows:
\begin{compactenum}[1.]
\itemnotoc Service hours and social or professional development credits count as 1 hour
\itemnotoc No more than 15 hours can be counted from a single event
\itemnotoc Only 8 total social or professional development credits  can be counted
\itemnotoc Hours must come from at least 3 different service events.
\end{compactenum}
\end{enumsubsection}
\section{Gift} At the end of each term, in appreciation for their significant contribution, each Prestigious Active will be given a fitting reward to be determined at the discretion of the Officer Corps. % F23 strike remaining:, including invitation to attend the initiation banquet at no cost, where he/she will receive special recognition for his/her achievement and dedication.

\section{Timing and Deadlines} Any requirements completed following initiation but before the end of term may be counted toward the current term or the following full term. In addition, any requirements completed between the winter and fall terms may be counted toward fall. 

\section{Attaining PA during Initiation Semester} An electee may count any hours beyond those needed for initiating toward achieving PA status. Electee PA requirements are the same as those for an active except that the total number of hours required is 33 (34 if no leadership credit is earned). Additionally, up to 10 social credits may be counted toward this number.
%\section{Substitution} Where circumstances merit, the Membership Officer may choose to allow any member to substitute one form of requirement for another, provided that the total number of hours completed is not diminished as a result of this. % Revised to Exceptions section during F23
\section{Exceptions} Where circumstances merit, the Membership Officer may choose to allow any member to substitute service hours for any form of requirement, with the exception of leadership credit, provided that the total number of hours completed is not diminished as a result of this. With consent of the Executive Committee, the Membership Officer may assign leadership credit to someone who has not otherwise earned it. 

%\section{Attaining PA during Summer} A member achieve PA status during the summer term. Summer PA requirements are the same as those for a full term except that the total number of hours required is 38. Additionally, up to 14 social credits may be counted toward this number.

%%NOT Officer Corps Changeable
\chapter{Officer Requirements and Descriptions}\label{sec:officerreq}
Descriptions for each officer position follow below. In addition to the specific duties of each office an officer is expected to complete at least four service hours, attend at least two Tau Beta Pi socials as well as both the initiation and the banquet.  These requirements are necessary for obtaining Distinguished Active status but are not sufficient.  To become a Distinguished Active an officer must also satisfy the requirements in Appendix~\ref{sec:DAstatus}. Appendix~\ref{sec:ExecComm}--\ref{sec:ChapterTeam} list only permanent officer members of each Team. Ad hoc officers and their team affiliation are listed in Appendix~\ref{sec:AdHocOfficers}.

\section{Executive Committee}\label{sec:ExecComm}
These officers serve as members of the Executive Committee.
\begin{enumsubsection}
\item{President} The President is the official representative, spokesperson, chief executive officer and chief operating officer of Tau Beta Pi Michigan Gamma. The President, with the assistance of Team Leads, ensures that each officer and chair is provided with a written list of all duties for which they are responsible and sees that they are fulfilled.  The President must also request a list of eligible candidates for election prior to the beginning of each term. The President must also prepare a meeting schedule at the beginning of the term and is responsible for scheduling rooms for the general meetings. The President serves as the lead for the Executive Committee. %F23 cause to be obtained --> request

\item{Vice President} The Vice President oversees all electee activities and business. The Vice President is responsible for ensuring that all electees complete the election requirements prior to initiating. The Vice President also assists the President with leadership and planning duties, and is the alternative representative of Tau Beta Pi in lieu of the President. The Vice President is also responsible for conducting the exemplary character process of prospective undergraduate members as specified in Appendix \ref{sec:ugradreqs}. The Vice President additionally serves as the lead for the Electee and Membership Team.

\item{Secretary} The Secretary is responsible for completing all paperwork and ensuring that it is submitted to the National Office on time, including but not limited to the Chapter Survey. The Secretary also serves as the main contact between the chapter and National Office. The Secretary holds the responsibilities of the Recording Secretary and is in charge of taking minutes at the meeting and making them available to the active membership, and of ensuring a timely transition of website permissions. % F23 removed: Additionally, the Secretary must fulfill the duties of the Cataloguer as specified in the National Bylaws 5.03.

\item{Treasurer} The Treasurer is  the chief financial officer of Tau Beta Pi Michigan Gamma and is responsible for sales, dues, chapter fundraising activities, budgeting, and reporting of the chapter's finances.  The Treasurer will approve or deny requests to use chapter funds, subject to appeal to the Advisory Board. The Treasurer must be an authorized signer on the chapter accounts.  Additionally, the Treasurer files annually (by May 15) IRS Form 990, 
 ``Return of Organization Exempt from Income Tax".  The Treasurer serves as a non-voting member of the Advisory Board described in Bylaw~\ref{sec:advbrd}. He/she may, with the consent of the Advisory Board, create and implement processes to assist in the budgeting and spending of the Chapter. The treasurer is additionally in charge of organizing, with assistance from the rest of the Executive Committee, the underclassmen mixer.

\item{Graduate Student Vice President} The Graduate Student Vice President (GSVP) is responsible for creating and tracking the electee progress of all graduate students and alumni candidates. The GSVP coordinates with the Vice President as needed. The GSVP sets up social events for graduate students and activities with other graduate student organizations. The GSVP conducts all meetings with graduate students and graduate student electees when necessary. The GSVP serves as a non-voting member of the Advisory Board described in Bylaw~\ref{sec:advbrd}.

\item{External Vice Presidents} The External Vice Presidents are responsible for the chapter's entire involvement with the Career Fair and the College of Engineering Honors Brunch. They chair the Tau Beta Pi awards committee. In the winter, they are in charge of planning the Career Fair and organizing the Honors Brunch. In the fall, they are in charge of running the Career Fair. They also serve as ex officio members of the Professional Development Committee, though they are overseen by the President.
\end{enumsubsection}

%
%\section{Professional Development Committee}\label{sec:PDTeam}
%These officers serve as members of the Professional Development Committee.
%\begin{enumsubsection}
%\item{Professional Development Officer} The \st{Corporate Relations} \hl{\textbf{Professional Development}} Officer serves as the Chair for the Professional Development Committee and is  responsible for the relations of the chapter to corporate representatives. He/she is in charge of planning the topics for the MLK Luncheons. He/she is also  in charge of scheduling speakers for the meetings and luncheons. 
%\item{External Vice Presidents} The External Vice Presidents are responsible for the chapter's entire involvement with the Career Fair and the College of Engineering Honors Brunch. They chair the Tau Beta Pi awards committee. In the winter, they are in charge of planning the Career Fair and organizing the Honors Brunch. In the fall, they are in charge of running the Career Fair. They also serve as ex officio members of the Professional Development Committee, though they are overseen by the President.
%\end{enumsubsection}

\section{Events Team}
These officers serve as members of the Events Team.
\begin{enumsubsection}
\item{Service Coordinator} The Service Coordinator oversees all service projects and must ensure that all project leaders complete their projects on time. The Service Coordinator is also responsible for informing the electees and actives of all the projects and providing them with opportunities to sign up for the projects of their choice. Additionally, the Service Coordinator will plan and coordinate new service projects and assist with the preparation of the Chapter Survey. The Service Coordinator serves as the lead for the Events Team. 

\item{K-12 Outreach Officers} The K-12 Outreach Officers seek and coordinate opportunities for engineering outreach within the community.  They serve as the liaisons for Chapter-sponsored K-12 outreach programs within both the Chapter and the community. This includes the chapter-run MindSET program. 

\item{Campus Outreach Officer} The Campus Outreach Officer seeks and coordinates opportunities for outreach to College of Engineering students, including, but not limited to, coordinating a tutoring program and promoting outreach programs to the engineering campus. The Campus Outreach Officer is also in charge of planning the topics for the MLK Luncheons.

\item{Activities Officer} The Activities Officer oversees all chapter social events, including intersociety events, and ensures that any social project leaders complete their events on time. The Activities Officer will make arrangements for the initiation banquet. 

\item{Professional Development Officer} The Professional Development Officer serves as the Chair for the Professional Development Committee and is charged with overseeing the Chapter's professional development programming. The Professional Development Officer will seek to provide diverse programming for the Chapter and the College community. The Professional Development Officer is also responsible for managing the chapter's corporate relations.  
\end{enumsubsection}
% He/she is responsible for providing food and beverages at all general body meetings specified in Bylaw III.

\section{Chapter Team}\label{sec:ChapterTeam}
These officers serve as members of the Chapter Team.
\begin{enumsubsection}
\item{Chapter Development Officer} The Chapter Development Officer is responsible for working with the Executive Committee to investigate and carry out ways of pursuing new opportunities or improvements for the chapter. The Chapter Development officer additionally is responsible for the planning and execution of New Initiatives meetings each semester, which are described in Bylaw~\ref{sec:NImeetings}. The Chapter Development Officer serves as the lead for the Chapter Team. 

\item{Historian} The Historian is responsible for the generation and distribution of all chapter publications, including the chapter newsletter, which is named ``The Cornerstone", the Alumni Newsletter, and any website publications deemed necessary. The Historian is in charge of picture taking for the Chapter Survey and maintains chapter records, including member demographics, as needed. 

\item{Publicity Officer} The Publicity Officer is responsible for all internal and external chapter publicity, including the weekly announcements, social media presence (e.g. Facebook, Twitter), college-wide announcements, flier generation, and any other publicity deemed necessary.

\item{Membership Officer} The Membership Officer is responsible for keeping track of all MI-$\Gamma$ members after they have been initiated. This includes determining who achieves active,  distinguished active, and prestigious active  status ,tracking meeting attendance, maintaining the email lists, and handling all alumni relations, except the publication of the Alumni Newsletter.  The Membership Officer is also responsible for providing food and beverages at all general body meetings specified in Bylaw~\ref{sec:genmeetings}.
\end{enumsubsection}
%
%\section{The Social Team}
%These officers serve as members of the Social Team.
%\begin{enumsubsection}
%\item{Activities Officer} The Activities Officer makes arrangements for all chapter social events. He/she is responsible for providing food and beverages at all general body meetings specified in Bylaw III. He/she will make arrangements for the initiation banquet. The Activities Officer serves as the lead for the Social Team.
%
%\item{Intersociety Officer} The Intersociety Officer is responsible for all intersociety activities, including but not limited to intramural sports and intersociety socials.
%\end{enumsubsection}
%
%\section{The Electee and Membership Team}\label{sec:ElecteeTeam}
%These officers serve as members of the Electee and Membership Team. The Vice President is the lead of this team, but is listed under Executive Committee in Appendix~\ref{sec:ExecComm}.
%\begin{enumsubsection}
%
%\end{enumsubsection}%vspace{.2in}

%%Officer Corps Changeable
\section{Ad Hoc Officers}\label{sec:AdHocOfficers} The ad hoc officers and their descriptions follow:\\
\begin{enumsubsection}

%%\item{Summer Committee Chair---Expires August 30, 2014} \emph{Active May 1--August 30}	This position will serve as acting president during its term. It should be filled by the Fall President, if possible. Otherwise it should be filled by a former president or advisor. He/she will coordinate the local activities of the chapter during the summer months, head the Summer Committee, and will designate any needed vice-chairs.
%%\item{District 7 Conference Chair--Expires April 2013} shall coordinate the planning and running of the Tau Beta Pi District 7 Conference during any academic year in which MI-Gamma is hosting. He/she will additionally chair the District Conference Committee, which will assist in his/her duties. In years when MI-Gamma is not hosting a Disctrict 7 Conference, this position shall be left vacant. 
%\item{Convention Arrangements Officer -- Expires December 31, 2017} \removed{will be on the executive team and work with the officers, advisors, and headquarters to help plan and execute the 2017 National Convention. This officer will serve the Calendar year 2017 and is required to attend the Convention.} 

%\item{Website Officer}  \label{officer-website} The Website Officer, a member of the Chapter Team, is responsible for maintaining the website by making any changes as well as documenting the structure of the website so that other members can better understand how the website works. The Website Officer will also update the website according to the changes that members propose, as the officer sees fit. The Website Officer will chair the Website Committee with any Website Chairs, will work with the Website Chair(s) to manage the Website Committee, and will decide who needs administrator privileges for the chapter website with the officers and advisors. % Added 2/5/23.

\item{Website Officer} shall  be  responsible for maintaining the Chapter's website,   including coordinating the development of new capabilities and in documenting the process for members to contribute to the website's development. In the fulfillment of these duties, this officer will serve as a website chair and will co-chair the Website Committee (\ref{com-website}) with any other website chairs. The officer will further be responsible for granting elevated access, including administrator privileges, for the chapter website. This officer will be elected each semester and shall be a member of the Chapter team. This position will cease to exist at the end of the Fall 2023 semester. \label{officer-website}  % Added F23


\end{enumsubsection}


%%NOT Officer Corps Changeable
\chapter{Committees}\label{sec:committees}%Flesh this out with a section for the standing and a section for the ad hoc
\section{Committee Participation}\label{sec:commPart} At the beginning of each term, the Executive Committee, with input from the respective committee chairs, will establish guidelines for levels of involvement for each committee, including a minimum level of involvement necessary for gaining a leadership credit.
\section{Standing Committees}\label{sec:standingCommittees} The standing committees and their descriptions are as follows:
\begin{enumsubsection}
	\item{Professional Development Committee} The Professional Development Committee will consist of the Professional Development Team, and any other chapter members as determined by the Professional Development Officer, who chairs the committee. The committee is responsible for hosting Engineering Futures sessions, information sessions with companies, and other professional development events.
	\item{Website Committee} \label{com-website} The Website Committee is chaired by the Website Chair, includes members as determined by the Website Chair, and is responsible for maintaining of the chapter website and any supporting technology. Administrator privileges for the chapter website are only to be granted to officers, advisors, or committee members.
	%\item{Book Swap Committee} The Book Swap Committee is chaired by the Service Coordinator, who may appoint co-chairs as appropriate. The Committee is responsible for the planning and execution of the semesterly Book Swap.
	\item{Group Leaders Committee} The Group Leaders Committee is chaired by the Vice President, who will select members as appropriate. The committee is composed of all electee group leaders and is responsible for coordinating the electee group aspect of the election process.
  \item{Diversity, Equity, \& Inclusion  Steering Committee} This committee will be be chaired by the Diversity, Equity, and Inclusion Chairs and will serve to assist them in coordinating the DEI activities of the chapter. This committee will consist of a maximum of 2 members of the officer corps, excluding advisors, and a minimum of 3 other members including initiated members and candidates / electees. The committee will work with the executive team and the advisory board to refine and develop the chapter's objectives with regard to Diversity, Equity, and Inclusion as well as to implement such objectives. The committee will submit a report of its activities and progress to the chapter via the Cornerstone at least 2 times per semester.
  \item{K-12 OUTREACH COMMITTEE} This committee, which shall be part of the Events Team, consists of the Service Coordinator and K-12 Outreach Officers, who shall co-chair the committee and may appoint additional committee members by majority vote of the three officers.  All chairs reporting to the K-12 officers shall serve as members of the committee.  The Committee will serve to assist the K-12 Outreach Officers in coordinating, executing, and transferring knowledge regarding the K-12 Outreach efforts of the chapter. This committee is responsible for training all volunteers at chapter-sponsored events in responsible conduct around minors as well as other requirements as set by the University or the chapter. Any chapter-sponsored event featuring youth requires the Committee pre-approval, and must be co-led by at least one member of the committee. This Committee must meet at least once per month, and publicly document their meeting minutes.
\end{enumsubsection}

%%Officer Corps Changeable
\section{Ad Hoc Committees}\label{sec:AdHocCommittees} The ad hoc committees and their descriptions follow:\\
\begin{enumsubsection}
\item{Banquet Arrangements Committee} The Banquet Arrangements Committee consists of the Activities Officer, who chairs the committee, the Banquet Arrangements Chair(s), and any other active members or candidates as determined necessary by these members. This committee is responsible for organizing the end-of-semester banquet, including communicating with the venue, determining and managing the attendee signup process, publicizing the event to prospective attendees, and generating content to be presented at the banquet. This committee shall be part of the Events Team. % Added F23


%\item{DIVERSITY INCLUSION AND EQUITY STEERING COMMITTEE}  This committee will be be chaired by the Diversity, Equity, and Inclusion Chairs and will serve to assist them in coordinating the DEI activities of the chapter. This committee will consist of a maximum of 2 members of the officer corps, excluding advisors, and a minimum of 3 other members including initiated members and candidates / electees. The committee will work with the executive team and the advisory board to refine and develop the chapter's objectives with regard to Diversity, Inclusion, and Equity as well as to implement such objectives. The committee will submit a report of its activities and progress to the chapter via the Cornerstone at least 2 times per semester. This committee shall cease to exist on May 1, 2021.


%\item{Diversity Inclusion and Equity Coordination Committee --- Expires December 31, 2019} This committee will be be chaired by the Diversity, Equity, and Inclusion Coordinators and will serve to assist them in coordinating the DEI activities of the chapter. This committee will consist of a maximum of 4 members of the officer corps, excluding advisors, and a minimum of 4 other members including initaited members and candidates / electees. The committee will work with the executive team and the advisory board to refine and develop the chapter's objectives with regard to Diversity, Inclusion, and Equity as well as to implement such objectives. The committee will submit a report of its activities and progress to the chapter via the Cornerstone at least 3 times per semester.


%\item{K-12 Committee---Expires January 26, 2016}  This committee will be co-chaired by the K-12 Outreach Officers and will serve to assist them in coordinating the K-12 Outreach efforts of the chapter, including but not limited to the MindSET program.
%\item{K-12 OUTREACH COMMITTEE---Expires April 30 2017}  This committee consists of the Service Coordinator and K-12 Outreach Officers, who shall lead the committee and may appoint co-chairs additional committee members at their discretion. The Committee will serve to assist the K-12 Outreach Officers in coordinating the K-12 Outreach efforts of the chapter. This committee is responsible for training all participants at chapter sponsored events in responsible conduct around minors as well as other requirements as set by the University or the chapter. Any chapter sponsored event featuring youth on campus requires the pre-approval of the Committee, and must be co-led by at least one member of the committee.

%\begin{enumsubsubsection}
%	\item{Summer Treasurer} The summer treasurer will be responsible for filing and issuing reimbursements during the summer months and for keeping the Treasurer informed of the financial activity of the chapter.. Where possible this should be filled by the current or a former treasurer, otherwise it should be filled by someone with experience as an authorized signer. The Summer Treasurer should be designated as the fourth authorized signer for the SOAS account.
%	\item{Summer Service Coordinator} The summer service coordinator will be responsible for coordinating service projects over the summer, and for ensuring that the appropriate Project Reports are completed.
%	\item{Summer Social Chair} The summer social chair will coordinate socials over the summer months. This should be filled by the current or a former Graduate Student Coordinator when possible.
%	\item{Summer Secretary} During the summer months, the summer secretary is responsible for taking minutes at committee meetings, determining who achieves DA/PA status, and sending out announcements to members as needed.
%\end{enumsubsubsection}
%%\item{District 7 Conference Chair--Expires April 2013} shall coordinate the planning and running of the Tau Beta Pi District 7 Conference during any academic year in which MI-Gamma is hosting. He/she will additionally chair the District Conference Committee, which will assist in his/her duties. In years when MI-Gamma is not hosting a Disctrict 7 Conference, this position shall be left vacant. 

\end{enumsubsection}

\chapter{Chairs}\label{sec:Chairs} 
\section{Current Chair Positions} The Chairs and their descriptions follow:\\
\begin{enumsubsection}
\item{Website Chair} The Website Chair(s), along with the Website Officer (\ref{officer-website}) are responsible for updating and maintaining the chapter website, and serves as the chair of the website committee. This chair serves on the Chapter Team.  If there is no Website Officer, the responsibilities of the Website Officer fall on the Website Chair.
% amended February 2023
\item{IM Sports Chair} The IM Sports Chair is responsible for the chapter's participation in Intramural Sports and by default serves as the captain of any such teams. This chair serves on the Events Team.
\item{Outreach Chair} The Outreach Chair is responsible for assisting the
campus outreach officer in the outreach activities and events of the chapter. This chair serves on the Events Team.
\item{Apparel Chair} The Apparel Chair is responsible for the design and acquisition of any MI-G branded apparel that the chapter may desire. This chair serves on the Chapter Team.
\item{Alumni Relations Chair} The Alumni Relations Chair is responsible for developing and maintaining the chapter's alumni relations initiatives. The chair will work with the Secretary and membership officer to maintain an up-to-date MI-G alumni contact roster. This chair serves on the Chapter Team.
\item{Graduate Student Activities Chair} The Graduate Student Activities Chair(s) will work with the graduate student vice president to help plan and execute events for graduate student actives and candidates. Further, the chair(s) will support the graduate student vice president in the execution of the latter's duties. The chair(s) will be members of the Events Team but shall collaborate with the Graduate Student Vice President.
%\item{Convention Arrangements Chair} The Convention Arrangements Chair(s) will be on the executive team and work with the officers, advisors, and headquarters to help plan and execute the 2017 National Convention. This chair's term will be semesterly for Winter 2016 and Fall 2016.

\item{Banquet Arrangements Chair} The Banquet Arrangements Chair will work with the activities officer, the  service coordinator, and the executive team to plan and execute the chapter's semesterly end of term banquet. This chair serves on the Events team.

\item{Mindset Chair} The Mindset Chair(s) will work with the K-12 Officers and Service Coordinator to plan and execute MindSET events.  The chair(s) will be members of the Events Team.

%\item{New Initiatives Chair} The New Initiatives Chair will coordinate the Chapter's New Initiatives meetings including identifying topics, chairing the meeting, obtaining food, and summarizing discussion. The chair will work under the direction of the Chapter Development Officer. The chair will be a member of the chapter team and will exist for the Winter 2022 semester. % Added 1/15/22

\item{New Initiatives Chair} The New Initiatives Chair will assist with coordinating the Chapter’s New Initiatives meetings, including identifying topics, chairing the meeting, obtaining food, and summarizing discussion. The chair will work under the direction of the Chapter Development Officer. The chair will be a member of the chapter team  % Added F23

%\item{Interchapter Chair} The Interchapter Chair will coordinate service and social activities between the chapter and other TBP chapters, with a focus on those in District 7. This chair will be a member of the Events Team and will exist for the Fall 2018 semester. 

%\item{Graduate Student Speaker Series Chair} The Student Speaker Series Chair(s) will work with the Graduate Vice President, Service Coordinator, and Professional Development Officer to plan and execute the TBP Graduate Student Speaker Series. The Chair(s) will be members of the events team, and will exist for the Fall 2018 semester.

%\item{Graduate Student Speaker Series Chair}  The Student Speaker Series Chair(s) will work with the Graduate Vice President, Service Coordinator, and Professional Development Officer to plan and execute the TBP Graduate Student Speaker Series. The Chair(s) will be members of the events team, and will exist for the Fall 2020 semester.

% \item{Honors Brunch Chair} The Honors Brunch Chair will  assist the External Vice Presidents in coordinating the annual college of engineering honors brunch including the selection and scheduling of award committees. The Chair(s) will be members of the events team, and will exist through the Winter 2019 semester.

%\item{Diversity, Equity, and Inclusion Coordinator(s)} The Diversity, Equity, and Inclusion Coordinator(s) will assist the executive team and advisory board in developing, refining, and implementing Diversity, Equity, and Inclusion related policies, programs, and activities within the chapter and in the broader community. The Diversity, Equity, and Inclusion Coordinator(s) will chair the Diversity, Equity and Inclusion Coordination Committee and will be a member of the chapter team and will report jointly to the Chapter Development Officer and the President.

%\item{Diversity, Equity, and Inclusion Chair(s)}  The Diversity, Equity, and Inclusion Chair(s) will assist the executive team and advisory board in developing, refining, and implementing Diversity, Equity, and Inclusion related policies, programs, and activities within the chapter and in the broader community. The Diversity, Equity, and Inclusion chair(s) will lead the Diversity, Equity and Inclusion Steering Committee; will be a member of the chapter team; and will report jointly to the Chapter Development Officer and the President. This position will exist for the Winter 2021 semester.

\item{Merit Badge Day Chair} The Merit Badge Day Chair(s) will work with the K-12 Officers and Service Coordinator to plan and execute merit badge day events for boy scouts.  The chair(s) will be members of the Events Team. 

\item{Governing Documents Chair} The Governing Documents Chair(s), after being informed of changes by the President, will generate an updated version of the chapter Bylaws, Constitution, Financial Policy, and Strategic Plan immediately following all approved changes to these documents and submit the updated documents to the website committee to be uploaded to chapter’s website. The Governing Documents Chair(s) will serve on the Chapter Team. % Added F23


%\item{New Initiatives Chair} The New Initiatives Chair is responsible for running the chapter's New Initiatives Meetings, working with the Chapter Development Officer to set the agenda, and for providing such meeting food as is deemed appropriate. The New Initiatives Chair should be an officer or advisor and will additionally serve on the Chapter Team.
\end{enumsubsection}


%\chapter{Awards and Recognitions}\label{sec:awards}

%\section{Collegiate Awards} \added{The chapter sponsors the following awards which are awarded in conjunction with other prestigeous College of Engineering Awards and Recognitions at the annual honors brunch each March. The criteria for each award along with any corresponding monetary gift shall be managed by the College of Engineering with input from the External Vice Presidents and Treasurer.}

%\begin{enumsubsection}
%\item{First Year Student Award} \added{The External Vice Presidents select the recipient(s) of the First Year Student Award. This award is designed to honor one to three first year students for their academic achievement and exemplary character. Engineering students in the top $\frac{1}{8}$ of their freshman class are eligible to apply.}

%\item{The Tom S. Rice Tau Beta Pi Award} \added{The Tom S. Rice Tau Beta Pi Award recognizes an outstanding Tau Bate for their continued commitment to the pillars of the society including academic achievement and community leadership. Academic achievement will be considered broadly. Community leadership shall include, but is not limited to, college engagement, chapter involvement, and service. Applicants should embody the criteria of The Eligibility Code of the Tau Beta Pi Association.}
%\end{enumsubsection}

%\section{Chapter Awards} \added{The chapter routinely recognizes outstanding active members for their contributions to the realization of the ideals of the Tau Beta Pi Assocation and those of the Michigan Gamma chapter.}

%\begin{enumsubsection}
%\item{Outstanding Electee Awards} \added{The chapter may recognize an outstanding undergraduate and graduate student electee each term. Recipients shall stand-out compared to their peers. The Vice President and Graduate Student Vice President shall select an outstanding electee from their respective electee groups to receive this award.}
%\item{Outstanding Active Award} \added{The Membership Officer shall select an active Tau Bate who has gone above and beyond in their service to the chapter and the furthering of the ideals of the Tau Beta Pi Association.}
%\item{Outstanding Officer Award} \added{The President shall select an officer, with input from the officer corps, to receive the outstanding officer award. This award shal be given to the officer or officers who have differentiated themselves throughout their term in office.}

%\end{enumsubsection}

