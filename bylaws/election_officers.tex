\section{Officer Election Meeting}\label{sec:elecMeeting} Officers (with the exception of those listed in Bylaw~\ref{sec:appointedofficers}) are elected at the election 
of officers meeting scheduled between the second election of candidates meeting and the formal initiation ceremony each semester. Electees and active members are eligible to vote and run for office. Quorum for such a meeting is defined in Bylaw 5.07 of the Association.
\section{Active Status for Voting Meetings}\label{sec:elecQuorum} At the beginning of the official business of an election meeting:
\begin{enumsubsection}
\item*{Inactive Members} The Advisory Board may deactivate any member as specified in the  \href{http://www.tbp.org/off/ConstBylaw.pdf}{Tau Beta Pi Association Bylaws Section 6.06}; criteria include non-attendance at that meeting. Unless otherwise specified by the Advisory Board,
members so deactivated are automatically reactivated at the end of the meeting, or upon entering the meeting.
\item*{Alumni Members} Alumni members present, including those on the Advisory Board, may apply for active status.
\end{enumsubsection}
\section{Incumbent Eligibility} Any officer is eligible for election or re-election to any available office. An individual may only serve as the President for a total of two semesters.
\section{Nominations} Nomination of officers can be made from the floor at the time of the election.  Additionally, written nominations may be presented to the current officers prior to the elections and announced at the beginning of the election meeting.
\section{Election Order} The order of election must begin with President, Vice President, Graduate Student Vice President, Secretary (Winter term), and Treasurer (Fall term). After these elections, the default order is first the election of all remaining Team Leads, in the order presented in Appendix~\ref{sec:officerreq}, and then the remaining elected offices, again in the order presented in Appendix~\ref{sec:officerreq}. Following the four listed positions, the remaining order can be changed from the default by motion from the floor; however, a Team Lead must be elected prior to anyone else on that Team.
\section{Candidate Speeches} Each nominee to a contested office can give a speech to, and answer questions from, the general assembly but other contestants for the office will not hear either. All nominees are limited to 2.5 minutes (up to 1.5 for speech) except those for President,  Vice President and Graduate Student Vice President for whom the time limit is 5 minutes (up to 3 for speech), and the remaining Team Leads for whom the time limit is 3.5 minutes (up to 2 for speech). These limits may be extended for any office by motion from the floor, provided that the extension affect each candidate equally. Remote or proxy speeches are allowed, provided that the officers have been notified at least 24 hours before the start of the meeting of a candidate's desire to provide such a speech. Proxies may not take questions on behalf of a candidate.
\section{Candidate Discussion} Following all speeches for an office, the assembly of present and voting members, excluding the outgoing President, will engage in discussion concerning the candidates.  Following their election, the incoming President may not take part in the discussion of the candidates. Incoming Team Leads may not take part in the discussion of candidates for positions on their respective Teams. This discussion must remain strictly confidential and is limited to matters pertaining to an individual's ability to successfully carry out the responsibilities of the office. 
\section{Winning the Election} The nominee that receives a plurality in the blind vote will be awarded the officer position.  In the case of a tie, a second vote will be held between the leading candidates; this will be continued until a nominee receives a majority.  In the event of a tie resulting in no leading candidates, another discussion of the candidates will be held. In all elections using manual voting, the votes must be counted by three individuals who must be either advisors or officers not pursuing election that semester. In all elections where electronic voting is to be used, the vote counts must be observed by a minimum of two individuals who must be either advisors or officers not pursuing election that semester.  In the event of an uncontested election, the candidate may be granted the officer position by voice vote. 

 
\section{Term Commencement} The new officers will take office on the day following the chapter's main initiation directly following their election.
\section{Vacant Offices} If any elected office becomes vacant between the regular elections, a special election will be held at the next general body meeting to fill any and all vacancies in these offices. If the remaining members of the Officer Corps determine it necessary to fill the vacancy prior to the next general body meeting, they may, by a majority vote, appoint an interim officer to serve until the next general body meeting or a meeting as determined by the Advisory Board. Vacancies in appointed offices will be filled in accordance with the ad-hoc officer appointment procedure as specified in Bylaw \ref{sec:adhocofficerapt}. The officer(s) elected or appointed at that time will serve for the remainder of the vacated term, and will enter their office at the conclusion of the general body meeting wherein the election was conducted or immediately after the vote at the conclusion of the appointment procedure. This procedure may be initiated as soon as an officer issues an advance notice of resignation to the President, even if the vacancy does not yet exist, and the selected replacement will take office immediately upon the existence of the vacancy. %Changed in F23 3rd actives from: If any office becomes vacant between the regular elections, a special election will be held at the next general meeting to fill any and all vacancies. The officer(s) elected at that time will serve for the remainder of the vacated term, and will enter their office at the conclusion of the general body meeting wherein the election was conducted. %is between the regular elections correct?

