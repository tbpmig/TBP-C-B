\chapter{Government of the Chapter}

\section{Governing Bodies}
Michigan Gamma shall be governed by the Chapter assembled in a Voting Meeting and by the Officer Corps. These two bodies shall control all matters and affairs pertaining to the chapter as a whole.

\section{Chapter Voting Meetings}
\begin{enumsubsection}
\item{Time and Participation} Chapter voting meetings must be held at least twice a semester to elect candidates for membership and to elect new officers. The chapter membership or officers may designate additional general meetings to be voting meetings, provided they are advertised as such not after the later of the previous general meeting or two weeks prior to the desired voting meeting. Any active member may attend and vote. Quorum for chapter activities is defined by C-VII, 6 of the Constitution of the Association.

\item{Agenda}
Prior to any voting meeting, the chapter president will prepare an agenda for the meeting and will cause it to be distributed.

\item{Chair}
The Advisory Board will select a chair for the official business portions of all voting meetings. This chair should be a current advisor or the chapter president.

\item{Officer Election}
The chapter membership will elect members of the Officer Corps as provided in this Constitution and any Bylaws adopted by the chapter.

\item{Bylaws}
The chapter membership may enact Bylaws for the chapter, at any voting meeting. The procedure for enactment, or for amendment of the amending procedure of such Bylaws, must follow the same requirements for amending this Constitution.

\end{enumsubsection}
\section{Officer Corps}
The Officer Corps will consist of those persons holding offices specified by this Constitution and the chapter Bylaws. They will serve without compensation for terms of lengths specified in this Constitution or the chapter Bylaws. No term may exceed 13 months except those of advisors whose terms cannot exceed four years. The members of the Officer Corps will be elected by the chapter membership at designated voting meetings. Chapter Bylaws may specify specific offices of the Officer Corps as appointed. If any positions are to be filled by appointment, the chapter Bylaws must designate a procedure for such appointments to follow.

\section{Offices} The chapter must maintain, at a minimum, those officers prescribed in C-VI,6 of the Constitution of the Association. Additional officers may be prescribed by the chapter Bylaws. The chapter Bylaws may additionally combine officers listed in C-VII,8 of the Constitution of the Association, provided that the necessitated responsibilities are still provided for.

\section{Officer Corps Responsibilities} The officer corps will be responsible for managing the day-to-day affairs of the chapter in accordance with the governing precedence of the chapter.  The duties of the individual officers are those usually performed by persons holding such offices and as prescribed by the Constitution and Bylaws of the Association and the Bylaws of the chapter.

\section{Vacated Offices} If any elected office becomes vacant between the regular elections, a special election will be held at the next general body meeting to fill any and all vacancies in these offices. If the remaining members of the Officer Corps determine it necessary to fill the vacancy prior to the next general body meeting, they may, by a majority vote, appoint an interim officer to serve until the next general body meeting or a meeting as determined by the Advisory Board. Vacancies in appointed offices will be filled in accordance with the ad-hoc officer appointment procedure as specified in the Chapter Bylaws. The officer(s) elected or appointed at that time will serve for the remainder of the vacated term, and will enter their office at the conclusion of the general body meeting wherein the election was conducted or immediately after the vote at the conclusion of the appointment procedure. This procedure may be initiated as soon as an officer issues an advance notice of resignation to the President, even if the vacancy does not yet exist, and the selected replacement will take office immediately upon the existence of the vacancy. % Updated in F23 3rd actives from: If a vacancy occurs on the Officer Corps subsequent to election, and/or subsequent to taking office, a special election will be held at the next general meeting to fill any and all vacancies. The officer(s) elected at that time will serve for the remainder of the vacated term. If the remaining members of the Officer Corps determine it necessary to fill the vacancy prior to the next general meeting, they may, by a majority vote, appoint an interim officer to serve until the next general meeting.

\section{Removal} Officers may be removed from office at the pleasure of the membership as provided for in the Bylaws or in the parliamentary authority (See Const. Art. XIV, Sec. 7) adopted by the Association. % Added in October 2017 to comply with CCI's ridiculous requirements for constitutions. This language comes directly from the proposed constitution of the association (Conv. 2017)

\section{Fiscal Year} The fiscal year of the chapter shall begin on January 1st and end on December 31st.



